\documentclass[a4paper,10pt]{article}
\usepackage[english]{babel}
\usepackage[latin1]{inputenc}
%\bibliographystyle{plain}
\usepackage{enumerate}
\usepackage{graphicx}
\usepackage{amsmath}
%\allowdisplaybreaks % to keep equations from hitting page numbers
% \usepackage{mathtools}
\usepackage{hyperref}
\usepackage{caption} 
\captionsetup[table]{skip=10pt}
% \usepackage{fullpage}
\usepackage{subfigure}
% \usepackage[normalem]{ulem}
% % \usepackage[framed]{mcode}	% Matlab-kode i dokumentet. Google "mcode"
\usepackage{placeins} % Holder bilder borte fra matlab-koden min! \FloatBarrier over og under
\usepackage{listings}
\usepackage{color}
\usepackage{textcomp}
\lstset{
	language=[Visual]C++,
	keywordstyle=\bfseries\ttfamily\color[rgb]{0,0,1},
	identifierstyle=\ttfamily,
	commentstyle=\color[rgb]{0.133,0.545,0.133},
	stringstyle=\ttfamily\color[rgb]{0.627,0.126,0.941},
	showstringspaces=false,
	basicstyle=\small,
	numberstyle=\footnotesize,
	numbers=left,
	stepnumber=1,
	numbersep=10pt,
	tabsize=4,
	breaklines=true,
	prebreak = \raisebox{0ex}[0ex][0ex]{\ensuremath{\hookleftarrow}},
	breakatwhitespace=false,
	aboveskip={1.5\baselineskip},
	columns=fixed,
	upquote=true,
	extendedchars=true,
 	frame=single,
}

\setlength{\parindent}{0in} % indent before new paragraph

% New definition of square root:
% Rename \sqrt as \oldsqrt
\let\oldsqrt\sqrt
% Define the new \sqrt in terms of the old one
\def\sqrt{\mathpalette\DHLhksqrt}
\def\DHLhksqrt#1#2{%
\setbox0=\hbox{$#1\oldsqrt{#2\,}$}\dimen0=\ht0
\advance\dimen0-0.3\ht0
\setbox2=\hbox{\vrule height\ht0 depth -\dimen0}%
{\box0\lower0.4pt\box2}}

% nice d in integrals/derivatives
\newcommand{\dd}{\mathrm{d}}

\newcommand{\bvec}[1]{\mathbf{#1}}
% % \newcommand{\stimes}{{\times}}
% \newcommand{\e}[1]{\ensuremath{\times 10^{#1}}}
% 
% \newcommand{\up}{\uparrow}
% \newcommand{\dw}{\downarrow}
% \newcommand{\stack}[2]{\array{c}{\scriptstyle #1}\\[-1.1ex]{\scriptstyle #2}\endarray}


\title{Progress report FYS4411 project}
\author{Filip Sund \\ filipsu@fys.uio.no}
\date{Spring 2013}

\begin{document}
\maketitle{}

%\setcounter{equation}{1}
I have done tasks 1a through 1c, and parts of 2a. I have not made a method for computing the onebody density and the charge density in task 1d, mostly because I don't know how to do it, and I haven't looked into it yet. I made a method that found the optimal variational parameters using Newton's method, but I have since rewritten my whole program, and have yet to re-implement this method. \\

I since rewritten my whole program, now using a Jastrow and Slater class, a Wavefunction class and a Hydrogen-like orbitals-class. I have not yet implemented the 2p wavefunction, so the program can't do Neon yet, but this should be easy to implement. I am still using the numerical derivatives to find the local energy, because I haven't implemented the closed form gradient or laplacian functions in the Jastrow and Slater classes. The Slater class is also currently using the slow/stupid method for calculating the Slater-wavefunction and the Slater-ratio:
\begin{lstlisting}
double Slater::wavefunction(const mat &r)
{
    /* For use in the numerical derivative, in the numerical local energy,
     * and in the temporary ratio function */

    mat slaterUP(N, N);
    mat slaterDOWN(N, N);

    for (int i = 0; i < N; i++)
    {
        for (int j = 0; j < N; j++)
        {
            slaterUP(i,j) = orbitals->wavefunction(i, r.row(j));
            slaterDOWN(i,j) = orbitals->wavefunction(i, r.row(j+N));
        }
    }

    return det(slaterUP)*det(slaterDOWN);
}

double Slater::getRatio()
{
    return wavefunction(rNew)*wavefunction(rNew) /
            (wavefunction(rOld)*wavefunction(rOld));
}
\end{lstlisting}

The program is running correctly in parallel using MPI, but I haven't implemented blocking yet. \\

My plan is to have a proper Slater class working before the end of easter, including proper ``smart'' updating of the inverse of the Slater determinant matrix, and closed for expressions for the gradient and laplacian ratios for both the Jastrow and the Slater classes. I'm a bit behind because I have been working a lot on FYS4460, and I have also had some trouble with the quantum mechanics.
% \newpage
% \input{report_code}
%\bibliography{references}
\end{document}

%\begin{figure}[h!]
	%\centering
	%\includegraphics[width =.30\textwidth]{bilder/a_exp_plot.eps}
	%\parbox{5in} {
		%\caption{
			%\small{
				%Plot of $e^{-4x}$ to find the appropriate limits for the integration.
			%}
			%\label{fig:a_exp_plot}
		%}
	%}
%\end{figure}

%\begin{figure}
	%\begin{center}
		%\makebox[\textwidth] {
			%\subfigure[] {
				%\label{fig:7a}
				%\includegraphics[width =.60\textwidth]{bilder/b_eigen0}
			%}
			%\subfigure[] {
				%\label{fig:7b}
				%\includegraphics[width =.60\textwidth]{bilder/b_eigen1}
			%}
		%}
		%\parbox{5in} {
			%\caption{
				%\small{
					%Bildetekst.
				%}
				%\label{fig:7}
			%}
		%}
	%\end{center}
%\end{figure}

% \begin{figure}
% 	\begin{center}
% 	\makebox[\textwidth] {
% 		\subfigure[] {
% 			\label{fig:ca1}
% 			\includegraphics[width =.60\textwidth]{bilder/c_stability0.01}
% 		}
% 		\subfigure[] {
% 			\label{fig:ca2}
% 			\includegraphics[width =.60\textwidth]{bilder/c_stability0.5}
% 		}
% 	}
% 	\makebox[\textwidth]
% 	{
% 		\subfigure[] {
% 			\label{fig:ca3}
% 			\includegraphics[width =.60\textwidth]{bilder/c_stability1.0}
% 		}
% 		\subfigure[] {
% 			\label{fig:ca4}
% 			\includegraphics[width =.60\textwidth]{bilder/c_stability5.0}
% 		}
% 	}
% 	\parbox{5in} {
% 		\caption {
% 			\small {
% 				Bildetekst.
% 			}
% 			\label{fig:ca}
% 		}
% 	}
% 	\end{center}
% \end{figure}
